\documentclass[9pt]{beamer}
\usepackage[sfdefault]{roboto}
\usepackage{styles/fluxmacros}
\usefolder{styles}
\usetheme[style=asphalt]{flux}
\usepackage{xcolor}
\usepackage{color}
\usepackage{amsmath}
\usepackage{amssymb}
\usepackage{graphicx}
\usepackage{latexsym}
\usepackage[T1]{fontenc}
\usepackage[utf8]{inputenc}
\usepackage{wrapfig}
\usepackage{siunitx}
\usepackage{times}
\usepackage{tikz}
\usepackage{verbatim}
\usepackage{multimedia}
\usepackage{hyperref}
\usepackage{thumbpdf}
\usepackage{%
    pgf,%
    pgfarrows,%
    pgfnodes,%
    pgfautomata,%
    pgfheaps,%
    pgfshade%
}
\usepackage{url}
\usepackage{empheq}
\usepackage{fancybox}
\usepackage{esint}
\usepackage{lipsum}
\usepackage{listings}
\usepackage{mathptmx}
\usepackage{helvet}
\usepackage{tikz}%
\usepackage{circuitikz}
\usepackage{csvsimple}
\usepackage{pgfplots}
\usepackage{multimedia}
\usepackage{proba}
\usepackage[absolute,overlay]{textpos}
\usepackage{bibunits}
\usepackage{tcolorbox}
\usepackage{qrcode}
%\usepackage[texcoord,%
%            grid,%
%            gridunit=mm,%
%            gridcolor=red!60,%
%            subgridcolor=black!60%
%            ]{eso-pic}
\usepackage{enumerate}
\usepackage[makeroom]{cancel}
\usepackage{epstopdf}
\usepackage{booktabs}
\usepackage{calc}
\usepackage{enumitem}
\usepackage{algorithmic,algorithm}
\setbeamercovered{transparent}
% Informations

\epstopdfsetup{outdir=./}
\mode<presentation>
{
    \useinnertheme{progressbar}
}
%\setbeamertemplate{items}[ball]
%~~~~~~~~~~~~~~~~~~~~~~~~~~~~~~~~~~~~~~~~~~~~-~~~~~~~~~~~~~~~~~~~~~~~~~~~~~~~~~~
%~~~~~~~~~~~~~~~~~~~~~~~~~~~~~~~~~~~~~~~~~~~~~-~~~~~~~~~~~~~~~~~~~~~~~~~~~~~~~~~
% Informations
\defaultbibliography{main}
%
\title{\Huge{Maximum likelihood estimation}\\
    \Huge{
        \textbf{
            for a stochastic SEIR system}
    }
}
\subtitle{%
        \textbf{
            \textcolor{gray}{
                 an application of the Grisanov's%
            }
       }
    \\    
    \textbf{    
        \textcolor{gray}{
            Theorem for the likelihood ratio
        }
    }    
    \\
    \normalsize{March 28, 2022}
}
%
\author{
    \normalsize{%
        CONACYT-UNISON-UNAM: FDV, FBL, SDIV
    }
}
%
\titlegraphic{assets/logo-unison.png}
%~~~~~~~~~~~~~~~~~~~~~~~~~~~~~~~~~~~~~~~~~~~~~~~~~~~~~~~~~~~~~~~~~~~~~~~~~~~~~~
\input{setup}
\newcolumntype{P}[1]{>{\centering\arraybackslash}p{#1}}
\begin{document}
    \titlepage
         \section*{Introduction}
    \part{Modeling with Noise}
        \section{Modeling with SDEs}
             \include{noise/Introduction/model_pertrubation}
        \section{Perturbation of parameters}
             \input{noise/ModelinWithNoise/noise_effects.tex}
             \input{noise/ModelinWithNoise/stochastic_modeling_alternatives.tex}
%             \input{noise/ModelinWithNoise/dtmc.tex}
%
%        \section*{Continuous Time Markov Chains (CTMC}
%             \input{noise/ModelinWithNoise/ctmc.tex}
%        \section*{SDEs}
%             \input{noise/ModelinWithNoise/sdes.tex}
        \begin{frame}
            \frametitle{Table of contents}
            \tableofcontents
        \end{frame}
        \section{Parameter estimation of SEIR-Covid-19 model based on a SDE}
	        \input{sto_mle_covid19/sto_mle_problem.tex}
            \begin{frame}{Stochastic perturbation}
    \begin{textblock*}{60mm}(10mm, 15mm)
        Perturbing the above deterministic base by Brownian Motion
    \end{textblock*}
%
    \begin{textblock*}{100mm}(20mm, 30mm)
        \begin{graybox}{
                $\mu dt\rightsquigarrow \mu dt + \sigma dW(t)$ gives
                our SDE SEIR-Covid-19
        }
            \begin{equation*}
                \begin{aligned}
                    d {S}(t)  =&
                    \big[\mu - \mu S(t) - f_{\beta} S(t)
                    +\gamma R(t)  \big] dt 
                    + \color{orange}{
                        \sigma \big(1- S(t)\big) dW(t)
                    }
                    \\
                    d {E}(t) =& \big[  f_{\beta} S(t)
                    - \kappa  E(t) - \mu E(t) \big] dt  
                    - \color{orange}{
                        \sigma E(t) dW(t)
                    }
                    \\
                    d {I_a}(t) =& \big[
                        p \kappa E(t)
                        -   (\alpha_a +\mu) I_a(t)  \big] dt  
                        - \color{orange}{
                            \sigma I_a(t) dW(t)
                        }
                    \\
                    d {I_s}(t) =& \big[
                        (1 - p) \kappa E (t)
                        - (\alpha_s +\mu)  I_s(t) \big] dt   
                        - \color{orange}{
                            \sigma I_s(t) dW(t)
                        }
                    \\
                    d {R}(t) =&
                        \big[
                        \alpha_a I_a(t) +
                        \alpha_s I_s(t) - (\mu + \gamma) R(t)
                        \big] dt
                        - 
                        \color{orange}{
                            \sigma R(t) dW(t)
                        },
                    \\
                    & t \in  [0, T] .
                \end{aligned}
            \end{equation*}
        \end{graybox}
    \end{textblock*}
\end{frame}
%------------------------------------------------------------------
\begin{frame}{Grisanov's likelihood ratio}
    Let $\mathbb{P}_{\beta, p}$ the law of solution to SDE. We use the
    following result
    \footnote{%
        S\"arkk\"a, Simo;
        Solin, Arno, Applied stochastic differential
        equations. Institute of Mathematical Statistics Textbooks,
        10. Cambridge University Press, Cambridge, 2019. ix+316 pp.
        ISBN: 978-1-316-64946-6
    }.%
    \begin{theorem}%
        [%
            {%
                Likelihood ratio of Itô processes
                S\"arkk\"a and Solin (2019, Thm. 7.4)%
            }%
    ]%
        Consider the It\^o processes
        \begin{equation*}
            \begin{aligned}
                dx =& f(x, t) + dB_t , \qquad x(0) = x_0,
                \\
                dy =& g(y, t) + dB_t, \qquad y(0) = x_0.
            \end{aligned}
        \end{equation*}
%        where $x(t), y(t) \in \mathbb{R}^D$ and the Brownian motion
%        $B_t \in \mathbb{R}^D$ has no singular diffusion matrix $\mathbf{Q}$.
        Then the ratio of probability laws of $\mathcal{X}_t$ and
        $\mathcal{Y}_t$ is given as
        \begin{equation*}
            \begin{aligned}
                \frac{p(\mathcal{X}_t)}{p(\mathcal{Y}_t)} =& Z(t),
                \\
                Z(t) =
                    \exp \Big(
                        -\frac{1}{2}
                        &
                        \int_0^t
                            [f(y, \tau) - g(y, \tau)]^{\top}
                            \mathbb{Q}^{-1}
                            [f(y, \tau) - g(y, \tau)] d \tau
                        \\
                        +
                        &\int_0^t
                        [f(y, \tau) - g(y, \tau)]^{\top}
                        \mathbb{Q}^{-1}
                        d B_{\tau}
                    \Big)
            \end{aligned}
        \end{equation*}
        in the sense that for an arbitrary functional $h(\cdot)$ of the path
        from $0$ to $t$,
        $$
            \EX{h(\mathcal{X}_t)}{} = \EX{Z(t) h (\mathcal{Y}_t)}{}
        $$
    \end{theorem}
\end{frame}
%-------------------------------------------------------------------------------
%-------------------------------------------------------------------------------%
\begin{frame}{The Lamperti transform}
    Suppose we have the SDE
    $$
        dX_t = a(t, X_t) dt + b(X_t) dW_t,
    $$
    where the diffusion coefficient depends only on the state variable.
    Such SDE can transform into one with unitary diffusion by applying the
    \emph{Lamperti} transform
    $$
        Y_t:= F(X_t) = \int_{z} ^ {X_t} \frac{1}{b(u)}du.
    $$
    Here $z$ is an arbitrary value and $Y_t$ solves
    $$
        d Y_{t} = 
            \left(
                \frac{a(t, X_t)}{b(X_t)}
                - \frac{1}{2} b_x (X_t)
            \right) dt
            + dW_t
    $$
    The results follows form the It\^{o} formula.
\end{frame}
%-------------------------------------------------------------------
\begin{frame}
    \begin{textblock*}{80mm}(0mm, 0mm)
        \begin{graybox}{Using It\^o and Lamperti transformations}
            \scalebox{.7}{%
                \parbox{\linewidth}{%
                    \begin{equation*}
                        \begin{aligned}
                            d {S}(t)  =&
                                \big[
                                    \mu - \mu S(t) - f_{\beta} S(t)
                                    + \gamma R(t)
                                \big] dt +                                                                
                                \sigma \big(1- S(t)\big) dW(t)
                            \\
                            d {E}(t) =&
                                \big[
                                    f_{\beta} S(t)
                                    - \kappa  E(t) - \mu E(t)
                                \big] dt  - \sigma E(t) dW(t)
                            \\
                            d {I_a}(t) =&
                                \big[
                                    p \kappa E(t)
                                    -(\alpha_a +\mu) I_a(t)
                                \big] dt  - \sigma I_a(t) dW(t)
                            \\
                            d {I_s}(t) =&
                                \big[
                                    (1 - p) \kappa E (t)
                                    - (\alpha_s +\mu)  I_s(t)
                                \big] dt
                                - \sigma I_s(t) dW(t)
                        \\
                            d {R}(t) =&
                                \big[
                                    \alpha_a I_a(t) +
                                    \alpha_s I_s(t) - (\mu + \gamma) R(t)
                                \big] dt
                                - \sigma R(t) dW(t),
                        \\
                        & t \in  [0, T] .
                        \end{aligned}%
                    \end{equation*}
                }
            }
            \tcblower
            $
                -\frac{1}{\sigma}
                d \mathbf{X}_{\beta,p}(t) =
                F\big(\mathbf{X}_{\beta,p}(t)\big) dt + d\mathbf{W}(t),
            $
        \end{graybox}
    \end{textblock*}
    %
    \begin{textblock*}{120mm}(10mm, 50mm)
        \scalebox{.7}{%
            \parbox{\linewidth}{%
    %
                \begin{equation*}
                    \mathbf{X}_{\beta,p}(t) :=
                    \begin{pmatrix}
                        \log \big(1-S(t)\big)
                        \\
                        \log \big(E(t)\big)
                        \\
                        \log \big(I_a(t)\big)
                        \\
                        \log \big(I_s(t)\big)
                        \\
                        \log\big (R(t)\big)
                    \end{pmatrix},
                    \quad
                    F\big(\mathbf{X}_{\beta,p}(t)\big):=
                    \begin{pmatrix}
                        \dfrac{\mu}{\sigma } -
                        \dfrac{f_\beta S(t) }{\sigma \big(1-S(t)\big)}
                        +
                        \dfrac{\gamma R(t)}{\sigma \big(1-S(t)\big)}
                        + \tfrac{1}{2} \sigma
                        \\
                        - \dfrac{f_\beta  S(t)  }{\sigma E(t)}
                        + \dfrac{ \kappa   }{\sigma }
                        + \dfrac{\mu }{\sigma }
                        + \dfrac{1}{2} \sigma
                        \\
                        - \dfrac{\kappa p E(t)}{\sigma I_a(t)}
                        + \dfrac{(\alpha_a + \mu )}{\sigma }
                        + \dfrac{1}{2} \sigma
                        \\
                        - \dfrac{\kappa (1-p) E(t)}{\sigma I_s(t)}
                        + \dfrac{(\alpha_s + \mu )}{\sigma }
                        + \dfrac{1}{2} \sigma
                        \\
                        - \dfrac{\alpha_a I_a(t) +
                            \alpha_s I_s(t)}{\sigma
                        R(t)}
                        + \dfrac{\mu + \gamma}{\sigma }
                        + \dfrac{1}{2} \sigma
                    \end{pmatrix},
                    \quad
                    d\mathbf{W}(t):=
                    \begin{pmatrix}
                        dW(t)
                        \\
                        dW(t)
                        \\
                        dW(t)
                        \\
                        dW(t)
                        \\
                        dW(t)
                    \end{pmatrix} .
                \end{equation*}
            }
        }
    \end{textblock*}
\end{frame}


            \input{sto_mle_covid19/mle.tex}
        \section{Final comments}
            \begin{frame}{Towards Optimal Stochastic Control of Epidemics}
    \begin{Huge}
        \begin{itemize}
            \item Discrete time
            \item Closed Loop Policies
            \item Games
        \end{itemize}
    \end{Huge}
    \href{https://github.com/SaulDiazInfante/Beamer-xxxii-semana-unison-2022-AppliedMathWorkshop.git}{Git-Hub}
    \\
    \qrcode{https://github.com/SaulDiazInfante/Beamer-xxxii-semana-unison-2022-AppliedMathWorkshop.git}
\end{frame}
 %---------------------------------------------------------------------------
%    \begin{frame}[allowframebreaks]
%        \frametitle{Final Remarks}
%            
%        % \bibliographystyle{acm}
%        \bibliography{main.bib}
%    \end{frame}
\end{document}
