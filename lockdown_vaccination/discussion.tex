%!TEX root = main.tex

\paragraph{Statement of principal finding}
    This study illustrated the implications of applying 
combined strategies of lockdown and vaccination to mitigate the curse 
of COVID-19. Our numerical experiments suggest that a combined and 
well-balanced policy between the relaxing times of lockdown and 
vaccine rollout would improve the mitigation of COVID-19 symptomatic 
prevalence and deaths--but with a delicate balance, 
with the economic impact. We obtained lockdown policies that modulate the 
releasing (or holding) of individuals synchronized with and vaccine rollout 
speed as a strategy to mitigate the symptomatic prevalence. 
Our policies captured time windows where it is convenient to modify the 
speed of releasing (or confining) individuals and the rate of vaccine 
administration according to the symptomatic prevalence. 
Here we quantify this convenience by computing a cost functional defined 
by a linear contribution due to the burden of the 
disease\textemdash quantify in DALYs\textemdash and quadratic cost 
of the implementation.

\paragraph{Strengths and weakness of the study}
    %TOPIC SENTENCE
    \begin{CheckList}{Goal}
        \Goal{open}{Argument for Strengths}
            \begin{CheckList}{Task}
                \Task{done}{Piecewise optimal policies}
                \Task{done}{Practical}
                \Task{done}{Modulation between
                    relaxation and inclusion of 
                    population in lockdown}
            \end{CheckList}
    \end{CheckList}

    \paragraph[]{STRENGTHS}
	        We got optimal policies that are practical. 
        Whit practical, we mean those capture actions that can be
	    implemented in the real world. Because the policies rely on 
	    piecewise constant values for a given period (here we use 
	    days), we argue that they are more realistic than policies 
	    based on measurable functions defined on continuos time.
        The policies defined on continuos time could implicate
	    continuous changes of a given action. %
        For example, it results more realistic to administer a 
        fixed number of jabs along with a given date than following 
        a configuration of several doses that would change continuously 
        on the same date.
    
            Besides, we synchronized the trade of between
        lockdown-release, vaccine-rollout but balancing the 
        economic cost.

    \paragraph[]{WEAKNESS}
        \begin{CheckList}{Goal}
            \Goal{open}{Argument for Weakness}
                \begin{CheckList}{Task}
                    \Task{done}{Vaccine Multi-dose.}
                    \Task{done}{Protection only against severe symptoms.}
                    \Task{done}{What expect respect 
                        to the protection against transmission.
                    }
                \end{CheckList}
        \end{CheckList}
  
            One limitation was the assumption of only one vaccine. 
        In almost all countries the vaccination campaigns consider at 
        least two developments. For example, in Mexico the vaccine portfolio 
        (at the day of writing) relies on at least four developments. Moreover, 
        this vaccine portfolio includes developments that differs in the number 
        of required doses. For example, Pfizers require two doses while Cansino 
        Bio only demands one.
        
            Further, we do not face additional vaccine administration 
        requirements as the time between doses or logistics\textemdash 
        each development implies different protocols. Since the approved 
        vaccine's protective efficacy against transmissions of SARS-CoV2 
        remains under study, we do not consider this hypothesis in our
        formulation. We recognize that this parameter would play an 
        important role, which is plausible to consider in future formulations.

    \paragraph{Strengths and weakness in relation to 
        other studies, discussing important differences
        in results
    }
    \todo{citations}
    \paragraph{STRENGTHS}
        \begin{CheckList}{Goal}
            \Goal{open}{Argument for Strengths}
                \begin{CheckList}{Task}
                    \Task{done}{%
                        NPI's optimal Policies with non Vaccination
                        }
                    \Task{done}{%
                        and compare our results.}
                    \Task{done}{What expect respect 
                        to the protection against transmission.
                    }
                    \Task{done}{Optimal Control of Vaccination Rate
                    cite \cite{Libotte2020} among others  and compare our 
                    results
                    }
                \end{CheckList}
        \end{CheckList}
        
        Works like
    \cite{Perkins2020,Palmer2020,Djidjou2020,Asamoah2021,Nabi2021} 
    models NPI's for COVID-19 with optimal control. Some 
    contribution of this list  combines two or more strategies to
    mitigate prevalence and deaths due to SARS-CoV-2. 
    For example, the authors of %
    \cite{Nabi2021} formulates a fractional ODE to model strategies as public
    education, treatment, and management of asymptomatic cases.  
    In \cite{Asamoah2021} Assamoth et al. unfold, an exhaustive analysis
    of the economic cost related with the combined health protocol of 
    physical distancing, media advocacy, mask wearing, hand-washing, 
    lockdown and contact tracing.
    Similarly, \cite{Djomegni2021,Jiang2020,Ullah2020}
    reports optimal policies but with significant emphasis on other aspects 
    related to the spread dynamics of COVID-19. However, the mentioned works 
    developed guidelines based on continuous-time did not optimize or not 
    include vaccination. Our contribution complements these contributions 
    with policies that are piecewise-constant. 
	    
	   Here we argue that our policies give a precise sequence of actions that 
    are feasible for implementation. Further, we calculated the cost and 
    balanced its performance synchronized and balanced with the economic 
    implications.

    \paragraph{WEAKNESS: Static optimization }
        \begin{CheckList}{Goal}
            \Goal{open}{Argument for Weakness}
                \begin{CheckList}{Task}
                    \Task{done}{Allocation.
                        Comment about optimal allocation as a 
                        static optimization problem and cite
                        cite {Bubbar, Buckner, Moore2021}.
                    }
                \end{CheckList}
        \end{CheckList}
    
        On the other hand, this work's limitation was the lack of stratification
    across ages and risk groups. Because DALY's definition depends on this 
    stratification, and the productive sector of an economy is closely related 
    to the workforce, we might prioritize accordingly. However, we see that our 
    result could complement the prioritization policies of relevant works like 
    \cite{Bubar2021,Matrajt2020a,Buckner2020}. In comparison, optimal vaccine 
    prioritization strategies answer the question: Who to vaccine first, we 
    respond when intensifying lockdown (holding) release vaccine rollout. 
    Thus, our contribution would be complementary.

    \paragraph{Meaning of the study: possible explanations 
        and implications for clinicians and policymakers}

        New and more contagious variants of SARS-CoV-2 appeared, 
    and the COVID-19 vaccine supply would be scarce, slow, and 
    complicated for countries like Mexico. Thus we expect three 
    synchronized events: another COVID-19 wave, an
    intensification of the vaccine rollout, and other lockdowns. 
    Here we argued that the above strategies also must be 
    synchronized and consider a delicate balance with the 
    economic impact.

    \paragraph{Unanswered questions and future research}
        
        Despite that NPIs have been implemented in most countries to mitigate 
    COVID-19, these strategies cannot develop immunity. Thus, vaccination 
    becomes the primary pharmaceutical measure. However, this vaccine has to be 
    effective and well implemented in global vaccination programs. Each 
    development implies particular issues--like logistics number of doses, 
    secondary effects, etc. For example, Mexico is administrating vaccines from 
    Pfizer-BioNTech, AstraZeneca, CanSino-Bio, and Sputnik V from Russia. 
    Each of these vaccines implies different requirements for its management, 
    protects with different efficacies, and differs in its number of doses. 
    We believe that this complicated landscape has significant implications in 
    the design of health policies.

        Therefore, we must face new challenges in distribution, stocks, 
    politics, vaccination efforts, and other uncertainties.          
    