%!TEX root = main.tex
% 1 definition of the topic plus
%background

    After more than 600 days, we face the devastating consequences of the still
ongoing COVID-19 pandemic. SARS-CoV-2 variants arise, but strict lockdown
is economical-infeasible---particularly in low-income countries like  Mexico.
Despite, the world have now several vaccine developments to fight against this
illness, the uncertainty towards a state similar to our previous normality is
the only matter of fact. In Mexico, healthcare infrastructures are poor, the
national vaccines' rollout still is under progress and limited by the
continuous adjustments required for its administration. Further, despite more
virulent and dangerous coronavirus variants arise, restrictive lockdown
is economically infeasible. Meanwhile, kids and students under isolation need
to retake face-to-face activities to overcome the harder consequences.
Because the most effective strategies available to face this pandemic are
lockdown and vaccination, we suppose that synchronizing these strategies would
improve their mitigation response. Here, we aim to unfold this premise by an
optimal control problem.
%
% 2. Accepted state of the art plus
%problem to be resolved

    On July 2020, the Strategic Advisory Group of Experts (SAGE) on
Immunization Working Group on COVID-19 Vaccines from WHO made public a set of
questions to encourage and direct the efforts of modelling groups to address
the design of vaccination policies \cite{sage2020}. Much of the  resent
development of vaccination models for COVID-19 try to  answer that questions.
We see particularly important topics like the epidemiological aspects of
SARS-CoV-2 and COVID-19, the vaccine landscape and supply-uptake scenarios.
The prioritization of scarce supplies and the economic impact of the regarding
vaccination policies. The first advances answer questions like how many
doses to allocate to each different group according to risk and age to minimize
the burden of COVID-19 or when to intensify the administration doses according
to the Outbreak evolution. However, the modeling of combined strategies like
NPIs and vaccination are under development.

    Our research in this manuscript explores the effect of synchronized Lockdown
and Vaccination interventions, to mitigate the  propagation of COVID-19.
We claim that this combined strategy would improve the mitigation of the
current pandemic and also would protect the economic harsh that implies the
implementation of Lockdowns.

%3 Authors' objectives
    Since health services' response will be limited by the vaccine stock and
logistics costs, implementing in parallel NPIs is imminent. We focus on
formulating and studying via simulation a lockdown-vaccination system by
considering the already announced back to school and vaccines recently approved
by Mexico Health Council. We aim to design a dose administration
schedule subject to a given vaccine stock, that has to be administrated in a
given period, but synchronizing with the release of the isolated
individuals\textemdash as the student population. For this purpose, we
formulate an optimal control problem that minimizes the burden of COVID-19 in
DALYs \cite{WhoDALY}, the cost generated by running the vaccination campaign,
and economic damages due to lockdown.


%4 INTRODUCTION TO LITERATURE
%
    Among the related literature about the mentioned interventions, we see
relevant the following advances. The problem of whom to vaccinate first, when
the number of available shots is limited, has been transformed into an optimal
allocation problem of vaccine doses in \cite{Bubar2020,Matrajt2020,Buckner2020}.
Other papers modeling NPIs as \cite{Naraigh2020,Ullah2020} consider diminishing
of contact rates by reducing mobility or modulating parameters regarding the
generation of new infections by linear controls. Mandal et al. models the
lockdown-quarantine in \cite{Mandal2020} as a control signal, and the authors
of \cite{Weitz2020} propose a shield immunity. Libotte et al. also report in
\cite{Libotte2020} optimal vaccination strategies for COVID-19. Our study takes
the allocation for granted and modulates vaccination and lockdown-release rates
as a combined strategy. Much of recent research in COVID-19 vaccination
consider separately vaccination or lockdown policies, but not synchronized or
balanced with economical implications.

% 6. AUTHORS's CONTRIBUTION
    To the best of our knowledge, this manuscript reports the first piecewise
optimal control model which combine and synchronize lockdown and vaccination
strategies. One of the main features of our model is that we consider piecewise
constant control policies instead of general measurable control policies---also
called permanent controls. Generally,  the control policies in continuous time
are impractical after a sufficiently large time. Since the authorities have to
make different choices permanently, but over discrete periods of time, we
perceive it more natural to deploy piecewise policies. The optimal policies we
find are constant in each interval of time, and hence results more practical and
consistent with reality.

%7. AIM OF THE PRESENTED WORK
    The aim of this contribution is to model and explore scenarios when an
isolated population retake activities face-to-face and then becomes susceptible
to acquire SARS-CoV-2, and at the same time, the rollout of a given vaccine is
under progress. Thus, a decision maker has to balance the release of Lockdown
and the number of vaccines to maximize the mitigation of symptomatic cases and
Deaths related with COVID-19.

% 8 MAIN RESULTS / CONCLUSIONS
    Our simulations suggest that combined and well-synchronized policies of
lockdown-release and vaccination would lead to a better response in the
mitigation of prevalence and deaths due to COVID-19.

% OUTLINE
    After this brief introduction, in \Cref{sec:Covid19_spread}, we formulate
the basic spread model for COVID-19 and calibrate its parameters. Then,
\Cref{sec:vaccination_model} establishes the lockdown-vaccination model and
discusses the regarding reproductive number in \Cref{sec:reproductive_number}.
We describe in \Cref{sec:optimal_controlled} our optimal control problem.
The optimal policies we find, by solving numerically the optimal control
problem, are presented in \Cref{sec:numerical_experiments}.
We conclude with some final comments in \Cref{sec:discussion}.
