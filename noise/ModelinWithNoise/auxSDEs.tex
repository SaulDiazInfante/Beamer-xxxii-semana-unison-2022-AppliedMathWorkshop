put{auxSDEs.tex}
     %%%%%%%%%%%%%%%%%%%%%%%%%%%%%%%%%%%%%%%%%%%%%%%%%%%%%%%%%%%%%%%%%
    \begin{frame}{Formuling a SIS-SDE}
        Now, consider $\{I_t\}_{t \geq 0}$. Define
        \begin{align*} 
            & p(y, t + \Delta t; x ,t)
            \\
            & y = I_{t + \Delta t},
            \quad x =I_t
        \end{align*}
              With this notation. Set $\Delta i = 1$, then
        \begin{align*}
            \frac{d p_i}{dt} 
                =&
                p_{i -1} b(i_1) + p_{i + 1} d(i + 1)
                - p_i [ b(i) + d(i)]
            \\
                &=
                - \frac{
                    p_{i + 1}[d(i + 1) - d(i + 1)]
                    - 
                    p_{i - 1}[d(i -1) - d(i - 1)]
                }{2 \Delta i}
            \\
                & +
                \frac{1}{2}
                \frac{
                    p_{i + 1}[d(i + 1) + d(i + 1)]
                    -
                    2 p_i[b(i) + d(i)]
                    +
                    p_{i - 1}[d(i -1) + d(i - 1)]}{
                        (\Delta_i) ^ 2
                    }
        \end{align*}
    \end{frame}
    %%%%%%%%%%%%%%%%%%%%%%%%%%%%%%%%%%%%%%%%%%%%%%%%%%%%%%%%%%%%%%%%%%%%%%%%%%%%
    \begin{frame}{}
            Let $i = x$, $\Delta i = \Delta x$ and $pi(t) = p(x,t)$.
        Thus, letting  $\Delta x to 0$, we obtain the FKE
        \begin{align*} 
            \frac{\partial p(x,t)}{\partial t}
                =&
                    \frac{\partial}{\partial x} \{   
                            [b(x) - d(x)] p(x,t)
                        \}
                +
                    \frac{1}{2}
                        \frac{\partial ^ 2}{\partial t ^ 2} \{
                            b(x)  + d(x) p(x,t) 
                        \}
                \\
                =&
                    \frac{\partial}{\partial x}
                    \left \{
                        \left [
                            \frac{beta}{N}
                            x (N - x)
                            - (b + \gamma) x
                        \right]
                        p(x,t)
                    \right \}
                \\    
                &+
                    \frac{1}{2}
                    \frac{\partial ^ 2}{\partial x ^ 2}
                    \left \{
                        \left [
                            \frac{\beta}{N}
                                x (N -x) + (\beta + \gamma) x 
                        \right ] p(x, t)
                    \right \}
        \end{align*}
    \end{frame}
%%%%%%%%%%%%%%%%%%%%%%%%%%%%%%%%%%%%%%%%%%%%%%%%%%%%%%%%%%%%%%%%%%    
    \begin{frame}{}
        Using the SIS-CTMC probablity transition kernel 
        \begin{align*}
            &p_{ji}(\Delta t):=
                \begin{cases}
                    \frac{\beta i (N - i)}{N} \Delta t 
                        + o(\Delta t),     
                        & j = i + 1
                    \\
                    (b + \gamma) i \Delta t 
                        + o(\Delta t),
                        &   j = i - 1
                    \\
                    1 - \left [
                            \frac{\beta i (N - i)}{N} +
                            (b + \gamma) i %                
                        \right] \Delta t
                        + o(\Delta t) , 
                        & j=i
                    \\
                    o(\Delta t) & \text{otherwise}
                \end{cases}
        \end{align*}
        If we assume that increment $\Delta t$ of transition follows a exponential
        distribution  and is suficiently small.
        Results that increment 
        $$
            \Delta I = I{t + \Delta t} - I_t
        $$  
        has normal distribuition, with following expectation and variance.
        Fix time $t$ such taht $I_t = i$
        \begin{align*}
            \E{\Delta I}
                =&
                    b(I_t) \Delta t - d(I_t) \Delta t + o(\Delta t)
                \\
                &=
                    \underbrace{
                        [b(I)_t - d(I_t)]
                    }_{:=\mu(I_t)} + o(t) 
        \end{align*}
    \end{frame}    
%%%%%%%%%%%%%%%%%%%%%%%%%%%%%%%%%%%%%%%%%%%%%%%%%%%%%%%%%%%%%%%%%%%%%%%%%%%%
   \begin{frame}{}
        Thus
        \begin{align*}
            \VarX{\Delta I_t}
                &=
                    \EX{\Delta I_t ^ 2} - [\EX{\Delta I_t}] ^ 2
                \\
                & = 
                    \underbrace{
                        [b(I_t)  + d(I_t) ]
                    }_{:=\sigma^2(I_t)}
                     \Delta t + o(\Delta t)
        \end{align*} 
        Since 
        $
            \Delta I_t \sim \mathcal{N}
             (
                \mu(I_t) \Delta t, 
                \sigma ^ 2 (I_t) \Delta t
            )
        $,
        we see that 
        \begin{align*}
            I_{t+\Delta t} 
                &= 
                    I_t + \Delta I_t
            \\
                &\approx
                    I_t + \mu(I_t)\Delta t 
                    + \sigma(I_t) \sqrt{\Delta t} \eta
            \\
                & \eta \sim \mathcal{N}(0,1)
        \end{align*}
    \end{frame}
    The Euler-Maruyama's recurrence equation.
    \begin{frame}{}
    Further, becouse under this setting, the Euler-Maruyama converge.
    Letting $\Delta t \to 0$, we deduce our SIS-SDE:
    \begin{equation*}
        dI_t = \mu(I_t) + \sigma(I_t) dW_t
    \end{equation*}
    Sustituting, the notation for birth and death processes
    \begin{equation*}
        dI_t =
            \frac{\beta}{N} I_t(N - I_t)
            - (b + \gamma) I_t
            +
            \sqrt{
                \frac{\beta}{N} 
                    I_t (N - I_t)
                    +
                    (b + \gamma)I_t
            }
            dW_t
    \end{equation*}
\end{frame}